% Select language used in document (ngerman or english). Automatically
% generated text is translated accordingly.
% use \selectthesislanguage in body.tex to switch to default language

%\documentclass[ngerman, paper]{mmt} % use for seminar paper
%\documentclass[ngerman, reviewversion]{mmt} % use for anonymized review version 
%\documentclass[ngerman, bachelorthesis]{mmt} % use for Bachelor thesis
\documentclass[ngerman, masterthesis]{mmt} % use for Master thesis

\usepackage{mathptmx}
\usepackage{graphicx}
\usepackage{times}
\usepackage{subfig}
\usepackage{float}
\usepackage[utf8]{inputenc}
\usepackage{listings}
\usepackage{makecell}
\usepackage[toc,page]{appendix}
\usepackage{hyperref}
\hypersetup{
    colorlinks,
    citecolor=black,
    filecolor=black,
    linkcolor=black,
    urlcolor=black
}
\usepackage{breakurl}

\usepackage{amsmath}
\usepackage[autostyle,german=guillemets]{csquotes}

\usepackage{verbatim}

\newcommand{\detailtexcount}[1]{%
  \immediate\write18{texcount -merge -sum -q #1.tex > #1.wcdetail }%
  \verbatiminput{#1.wcdetail}%
}

% command for highlighting text that was generate with AI support
\newcommand{\aigen}[1]{\textcolor{blue}{#1}}

% moved to cls file. use \selectthesislanguage to switch to default language
%\usepackage[english,ngerman]{babel}

\usepackage{acronym}
\usepackage{abbrevs}
%% the following solves a bug in the abbrevs package, that adds an empty
%% space after the abbrev
\makeatletter
\renewcommand\maybe@space@{%
  % \@tempswatrue % <= this is in the original
  \maybe@ictrue % <= this is new
  \expandafter   \@tfor
    \expandafter \reserved@a
    \expandafter :%
    \expandafter =%
                 \nospacelist
                 \do \t@st@ic
  % \if@tempswa % <= this is in the original
  \ifmaybe@ic % <= this is new
    \space
  \fi
}
\makeatother
%%

\usepackage{color}
\definecolor{lightgray}{rgb}{.9,.9,.9}
\definecolor{darkgray}{rgb}{.4,.4,.4}
\definecolor{purple}{rgb}{0.65, 0.12, 0.82}
\lstdefinelanguage{JavaScript}{
  keywords={break, case, catch, continue, debugger, default, delete, do, else, false, finally, for, function, if, in, instanceof, new, null, return, switch, this, throw, true, try, typeof, var, let, const, void, while, with},
  morecomment=[l]{//},
  morecomment=[s]{/*}{*/},
  morestring=[b]',
  morestring=[b]",
  ndkeywords={class, export, boolean, throw, implements, import, this},
  keywordstyle=\color{blue}\bfseries,
  ndkeywordstyle=\color{darkgray}\bfseries,
  identifierstyle=\color{black},
  commentstyle=\color{purple}\ttfamily,
  stringstyle=\color{red}\ttfamily,
  sensitive=true
}

\lstset{
   language=JavaScript,
   backgroundcolor=\color{lightgray},
   extendedchars=true,
   basicstyle=\footnotesize\ttfamily,
   showstringspaces=false,
   showspaces=false,
   numbers=left,
   numberstyle=\footnotesize,
   numbersep=9pt,
   tabsize=2,
   breaklines=true,
   showtabs=false,
   captionpos=b
}

\usepackage[url=false,authordate,bibencoding=auto,strict,noibid,backend=biber]{biblatex-chicago}
\bibliography{bibliography}

%% Add configuration options
\include{_configuration}

%% Paper title.

\title{\titlename}

%% This is how authors are specified in the conference style

%% Author 
\author{ \authorname\\ \scriptsize \authormail \\ \scriptsize 
\ifmmtlanguagegerman FH Salzburg \else Salzburg University of Applied Sciences \fi
}

%% A teaser figure can be included as follows, but is not recommended since
%% the space is now taken up by a full width abstract.
%\teaser{
%  \includegraphics[width=1.5in]{sample.eps}
%  \caption{This can be a teaser image of the thesis.}
%}

%% Abstract section for paper format.
\abstract{
    \ifmmtlanguagegerman 
        \selectlanguage{ngerman}
        \input{kurzfassung}
    \else 
        \selectlanguage{english}
        %%%%% IMPORTANT: do not use the acronym package here, but only in the main text. I.e., define acronyms manually here for now. %%%%%

Lorem ipsum dolor sit amet, consectetur adipiscing elit. Aenean venenatis nulla vestibulum dignissim molestie. Quisque tristique tortor vitae condimentum egestas. Donec vitae odio et quam porta iaculis ut non metus. Sed fermentum mauris non viverra pretium. Nullam id facilisis purus, et aliquet sapien. Pellentesque eros ex, faucibus non finibus a, pellentesque eu nibh. Aenean odio lacus, fermentum eu leo in, dapibus varius dolor. Lorem ipsum dolor sit amet, consectetur adipiscing elit. Proin sit amet ornare velit. Donec sit amet odio eu leo viverra blandit. Ut feugiat justo eget sapien porttitor, sit amet venenatis lacus auctor. Curabitur interdum ligula nec metus sollicitudin vestibulum. Fusce placerat augue eu orci maximus, id interdum tortor efficitur.

    \fi
}

%%%%%%%%%%%%%%%%%%%%%%%%%%%%%%%%%%%%%%%%%%%%%%%%%%%%%%%%%%%%%%%%
%%%%%%%%%%%%%%%%%%%%%% START OF THE PAPER %%%%%%%%%%%%%%%%%%%%%%
%%%%%%%%%%%%%%%%%%%%%%%%%%%%%%%%%%%%%%%%%%%%%%%%%%%%%%%%%%%%%%%%%

\begin{document}
% TODO switch for english, german
\selectthesislanguage

\pagenumbering{gobble}

 % group open
\ifmmtpaper
    \begingroup 
    % is required because paper template messes with sizes
    \fontsize{12}{18}\selectfont        
    \setlength{\parindent}{0pt}
    \setlength{\parskip}{5pt plus 2pt minus 1pt}
    \sectionfont{\fontsize{14}{15}\selectfont}
\fi

\ifmmtreviewversion
    \titlename
\else
    \begin{titlepage}

% check if second advisor exists
\newcommand{\printsecondadvisor}[1]{%
  \ifcsname#1\endcsname%
  \ifmmtlanguagegerman Zweitbetreuer*in: \else Second Advisor: \fi \secondadvisor 
  \else%
    
  \fi%
}

\ifmmtmasterthesis

    
    % \begin{center}
    %     \Huge{ 
    %     	\textbf{\ifmmtlanguagegerman Masterarbeit \else Master Thesis \fi}
    %     }
    % \end{center}
    
    \newpage
    
    \thispagestyle{empty}
    
    \hfill \includegraphics[height=1.5cm]{images/FHSLogo.jpg}
    
    \vspace*{2cm}
    
    \Large{
    \titlename
    
    \vspace*{1cm}
    
    \ifmmtlanguagegerman
    Masterarbeit zur Erlangung des akademischen Grades
    \else
    Master thesis in partial fulfilment of the requirements for\\ the degree of 
    \fi
    
    \vspace*{0.5cm}
    
    \textit{Master of Science}
    }
    
    
    \vspace*{1.5cm}
    {\large
    \ifmmtlanguagegerman Autor*in: \else Author: \fi \authorname
    }
    \vfill
    
    {\normalsize
    \ifmmtlanguagegerman
    Vorgelegt am FH-Masterstudiengang MultiMediaTechnology, Fachhochschule Salzburg
    \else
    Submitted to the Master degree program MultiMediaTechnology, Salzburg University of Applied Sciences
    \fi
    
    
    \vspace*{1cm}
    
    \ifmmtlanguagegerman Betreuer*in: \else Advisor: \fi
    \advisor
    \\    
    \printsecondadvisor{secondadvisor}
    
    \vfill
    
    Salzburg, \ifmmtlanguagegerman Österreich, \else Austria, \fi  \thesisdate
    }

\else % Bachelor thesis title page

    \begin{center}
    
    \includegraphics[width=5cm]{images/FHSLogo.jpg}


    \vspace*{4cm}
    
    \fontsize{20.79}{18pt}{\selectfont        
    %\Large{
    	\textit{\textbf{\titlename}}
    %}
    }
    
    \vspace*{4cm}
    
    \fontsize{20.79}{18pt}{%\large{
    \ifmmtlanguagegerman
      \textbf{ \ifmmtpaper Seminararbeit \else Bachelorarbeit \fi }
    \else
        \textbf{ \ifmmtpaper Seminararbeit \else Bachelor Thesis \fi }
    \fi
    }
    
    
    \end{center}
    
    \vfill
    
    %\begin{tabular}{ll}
    \ifmmtlanguagegerman Autor*in: \else Author: \fi  \authorname  \\
    \ifmmtlanguagegerman Betreuer*in: \else Advisor: \fi \advisor \\
    Repository: \thesisrepo \\
    \printsecondadvisor{secondadvisor}
    
    Salzburg, \ifmmtlanguagegerman Österreich, \else Austria, \fi \thesisdate
    
    
    
    % uncomment the following 3 lines for an optional lock flag, max. 2 years!
    %\hfill
    %\color{red}
    %\framebox{Sperrvermerk bis 20/01/2012}

\fi

\end{titlepage}
       
\fi

    \onecolumn           
    
    \pagenumbering{roman}
    
\ifmmtpaper % does not need affidavit
\else \ifmmtreviewversion
      \else
        \newpage
        
\ifmmtlanguagegerman
\subsection*{Eidesstattliche Erklärung}


Ich erkläre hiermit eidesstattlich, dass  ich  die vorliegende Arbeit selbständig  und ohne fremde Hilfe verfasst, und keine  anderen  als die angegebenen Quellen und  Hilfsmittel benutzt  habe. Weiter versichere ich hiermit, dass ich   die den benutzten Quellen  wörtlich oder inhaltlich entnommenen Stellen als solche kenntlich gemacht habe.

Die Arbeit wurde bisher in gleicher oder ähnlicher Form keiner anderen Prüfungskommission weder im In- noch im Ausland vorgelegt und auch nicht veröffentlicht.

\else

\subsection*{Affidavit}

I herewith declare on oath that I wrote the present thesis without the help of third persons and without using any other sources and means listed herein; I further declare that I observed the guidelines for scientific work in the quotation of all unprinted sources, printed literature and phrases and concepts taken either word for word or according to meaning from the Internet and that I referenced all sources accordingly.

This thesis has not been submitted as an exam paper of identical or similar form, either in Austria or abroad and corresponds to the paper graded by the assessors.

\fi

\vspace*{3cm}

%{\bf \thesisdate{}}


\hfill

\ifmmtlanguagegerman
$\overline{Datum \hspace{2cm}}$ \hfill $\overline{{Unterschrift}\hspace{3cm}}$

\vspace*{1cm}

\hfill $\overline{{Vorname\hspace{2cm}Nachname}}$
 
 \else
 $\overline{Date \hspace{2cm}}$ \hfill $\overline{{Signature}\hspace{4cm}}$

\vspace*{1cm}

 \hfill $\overline{{First~Name\hspace{2cm}Last~Name}}$
 \fi

  % comment out for expose
      \fi
\fi
% group closing
\ifmmtpaper
\endgroup
\fi


\ifmmtpaper \else  % paper does not need this stuff
    
    \newpage
    \selectlanguage{ngerman}
    \section*{Kurzfassung}
    \input{kurzfassung}
    \ifmmtmasterthesis
    
    \vspace*{0.5cm} 
    \textbf{Schlüsselwörter:~} \keywordsgerman
    \fi
    \newpage
    \selectlanguage{english}
    \section*{Abstract}
    %%%%% IMPORTANT: do not use the acronym package here, but only in the main text. I.e., define acronyms manually here for now. %%%%%

Lorem ipsum dolor sit amet, consectetur adipiscing elit. Aenean venenatis nulla vestibulum dignissim molestie. Quisque tristique tortor vitae condimentum egestas. Donec vitae odio et quam porta iaculis ut non metus. Sed fermentum mauris non viverra pretium. Nullam id facilisis purus, et aliquet sapien. Pellentesque eros ex, faucibus non finibus a, pellentesque eu nibh. Aenean odio lacus, fermentum eu leo in, dapibus varius dolor. Lorem ipsum dolor sit amet, consectetur adipiscing elit. Proin sit amet ornare velit. Donec sit amet odio eu leo viverra blandit. Ut feugiat justo eget sapien porttitor, sit amet venenatis lacus auctor. Curabitur interdum ligula nec metus sollicitudin vestibulum. Fusce placerat augue eu orci maximus, id interdum tortor efficitur.

    \ifmmtmasterthesis
        
    \vspace*{0.5cm} 
    \textbf{Keywords:~} \keywordsenglish
    \fi
    \selectthesislanguage
    
    \newpage
    \tableofcontents
    
    \newpage
    \listoffigures
    \lstlistoflistings
    \listoftables
    \ifmmtlanguagegerman
\section*{Abkürzungsverzeichnis}
\else
\section*{Abbreviations}
\fi

% use as \ac{AR} in the text in body.tex and you don't have to think about
% introducing acronyms anymore correctly at the first mention in the text.
% do not use \ac in the Abstract/Kurzfassung. Otherwise, acronyms must be
% reintroduced in the main text
%
% if you want to use the acronym package for the paper template
% define the acronyms using acrodef and without the acronym environment
% \acrodef{AR}{Augmented Reality} somewhere in body.tex. this file
% will not be included when using the paper template.
\begin{acronym}
 \acro{AR}{Augmented Reality}
 \acro{ML}{Machine Learning}
\end{acronym}

    
\fi


\mmtcolumnmode % switch back to column formatting of stylesheet

\maketitle % used for paper formatting

\ifmmtpaper\else
\pagestyle{headings}
\fi
\pagenumbering{arabic}


%for reference to this section
\section{Introduction}
\label{section:Introduction}

This template is used for seminar papers, bachelor and master theses at MultiMediaTechnology of the Salzburg University of Applied Sciences. 

The structure of the template fits many theses works. Seminar papers often require their own structure as it is a literature review on a specific topic and does not present your own work.

Outline the research field and lead towards your research question. How is the investigated issue resolved in related work? What are limitations of these solutions? What is your contribution to find a solution?

\section{Related Work}
Introduce why this specific related work is important for your own work. Which areas do you cover and why? What do you take as inspiration and what do you do differently/improve upon? 

\section{System Overview}
Provide a high level overview of your system, approach, etc. 
Describe features, user interfaces, provide screenshots.
What does a user do with your application/system/interaction method?

\section{Implementation}
Provide implementation details such as the used software and our software architecture, highlight your own solutions to encountered difficulties. Describe relevant iterations of your implementation.

\section{Evaluation}
Describe your methodology. How did you evaluate your work? Why did you choose this methodology? Present results of your evaluation here.

\section{Discussion}
Discuss your results to answer your research question. Does your data support you hypotheses? Put your results into perspective by situating it in the research field/related work.

\section{Conclusion and Future Work}
Summarize your work, outline limitations and future work. 

% \section{Formatierung}
% \label{section:Formatting}

% Text mit beliebigen Sonderzeichen in UTF-8 ohne BOM \ldots
% ,
% \textbf{hervorgehobener Text},
% \texttt{void}\footnote{Fußnote 1},
% mathematische Formel im Text $\sum_{i=0}^n i^2$
% \ldots

% Referenz auf Unterabschnitt \ref{subsection:Coding} der Arbeit, automatisch richtig nummeriert.

% \textcite[]{Mulloni:2010} für einen einen Literaturverweis im laufenden Text.

% Literaturverweise sind essentiell für eine wissenschafliche Arbeit. \autocite[]{McConnell:2004:CCS:1096143}.

% Achtung: nur zitierte Literatur wird im Literaturverzeichnis
% angeführt.\footnote{Fußnote 2}


% Wir verwenden \LaTeX\footnote{ \url{http://en.wikibooks.org/wiki/LaTeX}} -- und das
% ist keine Quelle, sondern blos eine URL.

% \subsection{Figures machen was sie wollen}

% % h = try to place the figure Here
% % t = try to place the figure at the Top of a page
% % p = try to place this figure along with others on a separate Page
% % Note that LaTeX has a sophisticated ranking algorithm to place figures.
% % It is not always easy to accept LaTeX's placing but it is harder doing it
% % manually. Just let it go ;-)
% \begin{figure}[!ht]
% 	\centering
% 	\subfloat[Das Julia Fraktal]{
% 		\includegraphics[width=0.75\linewidth]{images/Julia-Fractal.png}
% 		%for reference of this subfigure only
% 		\label{subfigure:Julia-Fractal}
% 	}
% 	\qquad
% 	\subfloat[Noise für Tinteneffekte]{
% 		\includegraphics[width=0.75\linewidth]{images/Perlin-Coherent.png}
% 		%for reference of this subfigure only
% 		\label{subfigure:Perlin-Coherent}
% 	}
% 	\caption[
% 		Verschiedene Pixelgraphiken\newline
% 		% source url given in the table of figures
% 		\small\texttt{https://mediacube.at/wiki/}
% 	]{
% 		Verschiedene Pixelgraphiken
% 	}
% 	%for reference to all subfigures
% 	\label{figure:PixelImages}
% \end{figure}

% Unterstützte Pixelgraphikformate: PNG, JPEG, PDF.
% Angabe von height oder width meist wichtig.

% Referenz auf Abbildung \ref{figure:PixelImages} mit allen Teilbildern.
% Referenz auf Unterabbildung \ref{subfigure:Julia-Fractal}.

% %figure* stretches figure over both columns
% \begin{figure*}[t]
% 	\centering
% 	\includegraphics[width=0.9\textwidth]{images/KappaGamma.pdf}
% 	\caption{
% 		Vektorgraphik mit \LaTeX\ Beschriftung ($\kappa$, $\gamma$)
% 	}
% 	%for reference to this figure
% 	\label{figure:KappaGammaTau}
% \end{figure*}

% Referenz auf Abbildung \ref{figure:KappaGammaTau}.

% Bei Vektorgraphik mit \LaTeX\ Beschriftung keine Skalierung mit width oder height verwenden.

% Vektorgraphik mit \LaTeX\ Beschriftung kann etwa mit \texttt{ipe} erstellt werden.

% Unterstütztes Vektorgraphikformat: PDF. EPS muss konvertiert werden.


% \subsection{Unterabschnitt 2}
% %for references to this subsection
% \label{subsection:Coding}

% \begin{lstlisting}[
% 	label=listing:Main, %for reference to this listing
% 	float=h,
% 	caption=main.cpp,
% 	firstnumber=10
% ]
% int main(void) {
% 	while (true) {
% 	}
% 	return 0;
% }
% \end{lstlisting}

% Wie man in Listing \ref{listing:Main} in Zeile 10 sieht, kann man die Zeilennummern im Listing absichtlich setzen, hier z.B. auf 10. In Listing \ref{listing:closure} wurde davon nicht Gebrauch gemacht. In diesem Fall beginnt die Nummerierung bei 1.

% \begin{lstlisting}[
%     label=listing:closure,
% 	float=h,
% 	caption=Closure in Javascript,
% 	language=JavaScript
% ]
% function foo(x,y) {
%     let i = x;
%     return function(a) {
%         return i * 2;
%     }
% }
% \end{lstlisting}


% \subsubsection{Unterunterabschnitt i}

% Wörtliches Zitat:
% %select proper language if not in German
% \selectlanguage{english}
% \begin{quote}
% ``Erwin Unruh discovered that templates can be used to compute
% something at compile time. [...] The intriguing part of this exercise, however, was that the production of the prime numbers was performed by the compiler during the compilation process and not at run time.''

% \autocite[305]{Bosch2014}
% \end{quote}
% %select German again or the language that you were using before (note ngerman stands for New German)
% %\selectlanguage{ngerman}
% \selectthesislanguage


% \subsection{Unterabschnitt b}

% \begin{enumerate}
% 	\item Punkt 1
% 	\begin{enumerate}
% 		\item Unterpunkt 1
% 		\item Unterpunkt 2
% 	\end{enumerate}
% 	\item Punkt 2
% \end{enumerate}

% \begin{itemize}
% 	\item Punkt 1
% 	\begin{itemize}
% 		\item Unterpunkt 1
% 		\item Unterpunkt 2
% 	\end{itemize}
% 	\item Punkt 2
% \end{itemize}


% \subsection{Unterabschnitt c}

% \begin{table}[ht]
% 	\centering
% 	\begin{tabular}{r|rrr}
% 		    & $i$ & $j$ & $k$ \\ \hline
% 		$i$ &$-1$ & $k$ &$-j$ \\
% 		$j$ &$-k$ &$-1$ & $i$ \\
% 		$k$ & $j$ &$-i$ &$-1$
% 	\end{tabular}
% 	\caption{
% 		Multiplikationstabelle für Quaternionen
% 	}
% 	\label{table:Quaternions}
% \end{table}

% Referenz auf Tabelle \ref{table:Quaternions}.

% \section{Abschnitt 2}
% \label{section:MathematicalStuff}

% Sei $f(x)$ eine stetige Funktion, so ist die \textbf{Fourier Transformierte}
% $F(\omega)$ wie folgt definiert:
% \begin{equation}
% \label{equation:FourierDefinition}
% 	F(\omega) = \int_{-\infty}^{\infty} f(x) e^{-i\omega t} dt
% \end{equation}

% Referenz auf mathematische Gleichung (\ref{equation:FourierDefinition}).

% Unnummerierte Gleichung:
% \begin{equation*}
% 	e^{i\varphi} = \cos\varphi + i \sin\varphi
% \end{equation*}
% %you may also use \[ \] instead of \begin{equation*} and \end{equation*}

% Gleichungssystem:
% \begin{eqnarray}
% 	g(x) = f(x - x_0) & \Leftrightarrow &
% 		G(\omega) = F(\omega) e^{-i\omega x_0} \\
% 	g(x) = f(x) e^{i\omega_0 x} & \Leftrightarrow &
% 		G(\omega) = F(\omega - \omega_0)
% \end{eqnarray}
 % the main text

%\input{acknowledgements}

\ifmmtpaper

\printbibliography

\else % only use the following for thesis format

\newpage
\printbibliography

\fi


 % group open
\ifmmtpaper 
\begingroup 
    % is required because paper template messes with sizes
    \fontsize{12}{18}\selectfont        
    \setlength{\parindent}{0pt}
    \setlength{\parskip}{5pt plus 2pt minus 1pt}
    \sectionfont{\fontsize{14}{15}\selectfont}
\fi

\ifmmtpaper\else % nicht im paper

\newpage
\onecolumn
\begin{appendices}

\section{AI Methodology}
Document in this section, how AI was utilized to support the creation of this seminar paper or thesis. Which tools were used? Which steps of the thesis creation were supported (e.g., brain storming, literature search, quality analysis of papers, text generation, text summaries, paraphrasing, ...). If chat bots were utilized (e.g., ChatGPT), also provide key text prompts that influenced the thesis.

%\renewcommand{\thesubsection}{\Alph{subsection}}
\section{git-Repository}

According to the respective guidelines.

The repository must be uploaded to the MMT/HCI git server {\url{gitlab.mediacube.at}}

{\color{red}\url{https://gitlab.mediacube.at/fhs123456/Abschlussarbeiten-Max-Muster}}
	
\section{Vorlagen für Studienmaterial}

Study material if applicable. 

\section{Archived Websites}
% \show\UrlBreaks
\sloppy
\url{http://web.archive.org/web/20160526143921/http://www.gamedev.net/page/resources/_/technical/game-programming/understanding-component-entity-systems-r3013}, letzter Zugriff 1.1.2016

\url{http://web.archive.org/web/20160526144551/http://scottbilas.com/files/2002/gdc_san_jose/game_objects_slides_with_notes.pdf}, letzter Zugriff 1.1.2016

% DIESEN TEIL NICHT LÖSCHEN ODER ÄNDERN == BEGINN
%\fi % end if for the if \ifmmtreviewversion 
\end{appendices}
% DIESEN TEIL NICHT LÖSCHEN ODER ÄNDERN == ENDE


\fi

% group closing
\newpage

This work has the following word count (counted by texcount): 
%TC:ignore
\detailtexcount{body}
%TC:endignore

\end{document}
