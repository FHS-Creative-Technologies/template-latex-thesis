\begin{appendices}


\section{AI Methodology}

\textbf{You can delete the explanatory text in this section and add your AI documentation.}

Document in this section, how AI was utilized to support the creation of this seminar paper or thesis. Which tools were used? Which steps of the thesis creation were supported (e.g., brain storming, literature search, quality analysis of papers, text generation, text summaries, paraphrasing, spellchecking, ...). If generative AI was utilized (e.g., ChatGPT, CoPilot), also provide key text prompts that influenced the thesis. 

In the following, list each tool and provide a brief description how the tool as utilized. 

This section should have a maximum length of 2 pages.

\subsection{DeepL}
I used the tool (\url{https://addressoftool.com}) to [paraphrase|translate|...] text for this work.

\subsection{Copilot}
I used Copilot (\url{https://addressoftool.com}) to learn how to utilize RUST as a new programming language. .... 

\subsection{AI Tool 2}
... document tool usage ...

\subsection{Prompts}
I utilized the generative AI \url{https://bot.com} to create drafts for texts, support brainstorming, ...

... list exemplary prompts here ...

%\renewcommand{\thesubsection}{\Alph{subsection}}
\section{git-Repository}

According to the respective guidelines.

The repository must be uploaded to the MMT/HCI git server {\url{gitlab.mediacube.at}}

{\color{red}\url{https://gitlab.mediacube.at/fhs123456/Abschlussarbeiten-Max-Muster}}
	
\section{Vorlagen für Studienmaterial}

Study material if applicable. 

\section{Archived Websites}
% \show\UrlBreaks
\sloppy
\url{http://web.archive.org/web/20160526143921/http://www.gamedev.net/page/resources/_/technical/game-programming/understanding-component-entity-systems-r3013}, letzter Zugriff 1.1.2016

\url{http://web.archive.org/web/20160526144551/http://scottbilas.com/files/2002/gdc_san_jose/game_objects_slides_with_notes.pdf}, letzter Zugriff 1.1.2016

% DIESEN TEIL NICHT LÖSCHEN ODER ÄNDERN == BEGINN
%\fi % end if for the if \ifmmtreviewversion 
\end{appendices}
% DIESEN TEIL NICHT LÖSCHEN ODER ÄNDERN == ENDE
